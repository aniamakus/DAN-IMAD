\section{Wprowadzenie}
    \subsection{Cel ćwiczenia}
        Celem ćwiczenia było poznanie tzw. naiwnego klasyfikatora Bayesa oraz
        zbadanie i ocena jego działania na 3 określonych zbiorach danych. W trakcie
        badań należało uwzględnić różne metody dyskretyzacji danych i kroswalidacji
        oraz zaobserwować wpływ tych parametrów na wartości zadanych metryk.

    \subsection{Klasyfikator Bayesowski}
        Typowym zagadnieniem w uczeniu maszynowym jest zadanie klasyfikacji.
        Należy ono do grupy tzw. zadań uczenia nadzorowanego, czyli zakłada istnienie
        zbioru danych, w którym każda instancja (wpis, krotka danych) jest oznaczona
        odpowiednią etykietą (\textit{klasa}).

    \subsection{Dyskretyzacja}
    \subsection{Kroswalidacja}
    \subsection{Metryki}
    \subsection{Problemy}