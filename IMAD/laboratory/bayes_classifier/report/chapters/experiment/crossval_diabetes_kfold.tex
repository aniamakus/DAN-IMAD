Dla kroswalidacji zwykłej (K-Fold) można zauważyć, że dla wszystkich metod dyskretyzacji
wartości Accuracy były dość wysokie, przy czym już Recall osiągał znacznie gorsze
wyniki, tym bardziej dla metody CAIM. Widać zatem, że nie można jedynie polegać na
mierze jaką jest Accuracy, ponieważ inne kryteria jakości nauki klasyfikatora mogą być
niespełnione. Złe wyniki osiągane przez CAIM można wytłumaczyć poprzez małą liczbę kubełków
podczas dyskretyzacji wartości atrybutów.

Zatem zgodnie z ogólnymi zaleceniami powinno się stosować miarę F1, choć i ona
nie zawsze wysoka jej wartość oznacza dobry klasyfikator.

\begin{figure}[H]
    \center
    \includegraphics[width=\textwidth]{img/cv_scores_kfold/scoring_kfold_diabetes.png}
    \caption{Wykresy wartości metryk dla zbioru "Diabetes" -- kroswalidacja zwykła.}
\end{figure}

Związany z najlepszą wartością miary F1 został odczytany parametr kroswalidacji
równy $K = 9$ (patrz: Tabela 1). Dla tego parametru została również przedstawiona
macierz konfuzji (Rysunek 15). Jak widać w przypadku klasy oznaczającej chorą osobę
(True Label; Ill) klasyfikator nie najlepiej sobie radził z powierzonym zadaniem.
Z drugiej strony osoby, które faktycznie są zdrowe zostają w miarę dobrze klasyfikowane.

\begin{figure}[H]
    \center
    \includegraphics[width=0.6\textwidth]{img/conf_matrices/cm_Diabetes_cv9_KFold.png}
    \caption{Macierz konfuzji dla najlepszej wartości F1 -- kroswalidacja zwykła.}
\end{figure}

 \begin{table}[H]
    \center
    \caption{Wartości metryk dla zbioru "Diabetes" -- kroswalidacja zwykła.}
    \begin{tabular}{|c|c|c|c|c|c|c|c|c|c|}
        \hline
        \multirow{2}{*}{\textbf{Metoda dyskr.}} & \multirow{2}{*}{\textbf{Metryka}} & \multicolumn{8}{|c|}{\textbf{CV}} \\ \cline{3-10}
                                                &  & 2 & 3 & 4 & 5 & 6 & 7 & 8 & 9 \\ \hline
            \multirow{4}{*}{\textit{Brak}}  & Accuracy & 0.754 & 0.745 & 0.757 & 0.753 & 0.754 & 0.753 & 0.754 & 0.757 \\ \cline{2-10}
                                             & Precision & 0.663 & 0.648 & 0.668 & 0.662 & 0.663 & 0.661 & 0.665 & 0.667 \\ \cline{2-10}
                                             & Recall & 0.601 & 0.59 & 0.601 & 0.593 & 0.601 & 0.597 & 0.593 & 0.604 \\ \cline{2-10}
                                             & F1 & 0.63 & 0.617 & 0.633 & 0.626 & 0.63 & 0.627 & 0.627 & 0.634 \\ \hline  \hline


                                            \multirow{4}{*}{\textit{Equal-width}}  & Accuracy & 0.658 & 0.661 & 0.66 & 0.658 & 0.659 & 0.658 & 0.661 & 0.663 \\ \cline{2-10}
                                             & Precision & 0.514 & 0.521 & 0.519 & 0.514 & 0.516 & 0.514 & 0.521 & 0.524 \\ \cline{2-10}
                                             & Recall & 0.351 & 0.366 & 0.362 & 0.351 & 0.362 & 0.354 & 0.369 & 0.369 \\ \cline{2-10}
                                             & F1 & 0.417 & 0.43 & 0.426 & 0.417 & 0.425 & 0.419 & 0.432 & 0.433 \\ \hline  \hline


                                            \multirow{4}{*}{\textit{Equal-freq}}  & Accuracy & 0.727 & 0.728 & 0.734 & 0.734 & 0.736 & 0.734 & 0.733 & 0.725 \\ \cline{2-10}
                                             & Precision & 0.638 & 0.637 & 0.652 & 0.651 & 0.654 & 0.655 & 0.647 & 0.633 \\ \cline{2-10}
                                             & Recall & 0.5 & 0.511 & 0.511 & 0.515 & 0.515 & 0.504 & 0.519 & 0.507 \\ \cline{2-10}
                                             & F1 & 0.561 & 0.567 & 0.573 & 0.575 & 0.576 & 0.57 & 0.576 & 0.563 \\ \hline  \hline


                                            \multirow{4}{*}{\textit{CAIM}}  & Accuracy & 0.664 & 0.668 & 0.667 & 0.668 & 0.66 & 0.661 & 0.66 & 0.659 \\ \cline{2-10}
                                             & Precision & 0.581 & 0.614 & 0.615 & 0.61 & 0.566 & 0.574 & 0.561 & 0.556 \\ \cline{2-10}
                                             & Recall & 0.134 & 0.131 & 0.119 & 0.134 & 0.112 & 0.116 & 0.119 & 0.112 \\ \cline{2-10}
                                             & F1 & 0.218 & 0.215 & 0.2 & 0.22 & 0.187 & 0.193 & 0.197 & 0.186 \\ \hline  \hline
            \hline
        \end{tabular}
    \end{table}
