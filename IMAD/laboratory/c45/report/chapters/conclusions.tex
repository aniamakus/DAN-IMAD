\pagebreak

\section{Wnioski}
\begin{itemize}
  \item{Drzewa decyzyjne prezentują wyniki uczenia w sposób bardziej przystępny dla człowieka. 
        Można bardzo łatwo prześledzić proces podejmowania decyzji w wygenerowanym drzewie -- dla danej
        instancji należy wybierać odpowiednie gałęzie w zależności od wartości atrybutów instancji oraz
        warunków określonych w węzłach drzewa.}
  \item{Dzięki zastosowaniu różnych metod przycinania drzewa można ograniczyć głębokość drzewa i 
        pozwolić na lepszą generalizację zbioru danych.}
  \item{Dla zbioru danych "Diabetes" pozwalało otrzymać stosunkowo lepsze wyniki. Dla kroswalidacji zwykłej: 
    Precision równe 0.76 zamiast 0.66, Recall: 0.90 zamiast 0.6 oraz F1: 0.83 zamiast 0.63, natomiast Accuracy 
    było na takim samym poziomie (dane dla najlepszych wyników obu algorytmów względem miary F1). Zastosowanie
    kroswalidacji stratyfikowanej na tym zbiorze danych nie spowodowało dużych zmian i wyniki są porównywalne
    (prawie identyczne jak w przypadku kroswalidacji zwykłej).}
  \item{W przypadku zbioru danych "Glass" przy zastosowaniu kroswalidacji zwykłej, zastosowanie algorytmu
        drzewa C4.5 pozwalało uzyskać znacząco lepsze wyniki -- prawie 2 razy lepsze (patrz: Tabela 12).
        Natomiast przy zastosowaniu kroswalidacji stratyfikowanej różnice nie były tak duże; należy również pamiętać,
        że w tej sytuacji klasyfikator Bayesowski (z metodą dyskretyzacji CAIM) uzyskał również lepsze wyniki
        niż dla kroswalidacji zwykłej.}
  \item{Dla zbioru danych "Wine" algorytm C4.5 uzyskał nieznacznie gorsze wyniki (patrz: Tabela 16). Należy jednak
       zauważyć, że oba algorytmy uzyskiwały, dla tego zbioru danych, bardzo dobre osiągi (rzędu 90\%).} 
  \item{Wpływ opcji konfiguracyjnych (parametrów) algorytmu C4.5 jest uzależniony od zbioru danych. Dla zbioru
    "Diabetes" zróżnicowanie wyników osiąganych przez algorytm jest dość małe w zależności od paramterów.
    Największe zróżnicowanie pojawia się w przypadku zbioru "Glass".}

\end{itemize}
