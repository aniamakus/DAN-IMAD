\pagebreak
\subsection{Wyniki kroswalidacji}
    Poniżej zostały przedstawione wyniki zastosowania kroswalidacji (z parametrem
    w postaci liczby podzbiorów; zmieniający się od 2 do 9 ze skokiem 1) zbiorów danych,
    które:
    \begin{itemize}
      \item{}
    \end{itemize}
    a następnie w ramach danego procesu kroswalidacji, wyznaczono wartości miar
    oceny jakości klasyfikatora. Dodatkowo zostały zamieszczone tabelki z dokładnymi
    wartościami tych miar.

    \subsubsection{Zbiór danych -- "Diabetes"}
        \begin{figure}[H]
    \includegraphics[width=\textwidth]{img/cv_scores_kfold/scoring_kfold_diabetes.png}
    \caption{Wykresy wartości metryk dla zbioru "Diabetes" -- krowalidacja zwykła.}
\end{figure}

 \begin{table}[H]
    \center
    \caption{Wartości metryk dla zbioru "Diabetes" -- krowalidacja zwykła.}
    \begin{tabular}{|c|c|c|c|c|c|c|c|c|c|}
        \hline
        \multirow{2}{*}{\textbf{Metoda dyskr.}} & \multirow{2}{*}{\textbf{Metryka}} & \multicolumn{8}{|c|}{\textbf{CV}} \\ \cline{3-10}
                                                &  & 2 & 3 & 4 & 5 & 6 & 7 & 8 & 9 \\ \hline
            \multirow{4}{*}{\textit{Brak}}  & Accuracy & 0.754 & 0.745 & 0.757 & 0.753 & 0.754 & 0.753 & 0.754 & 0.757 \\ \cline{2-10}
                                             & Precision & 0.663 & 0.648 & 0.668 & 0.662 & 0.663 & 0.661 & 0.665 & 0.667 \\ \cline{2-10}
                                             & Recall & 0.601 & 0.59 & 0.601 & 0.593 & 0.601 & 0.597 & 0.593 & 0.604 \\ \cline{2-10}
                                             & F1 & 0.63 & 0.617 & 0.633 & 0.626 & 0.63 & 0.627 & 0.627 & 0.634 \\ \hline  \hline


                                            \multirow{4}{*}{\textit{Equal-width}}  & Accuracy & 0.658 & 0.661 & 0.66 & 0.658 & 0.659 & 0.658 & 0.661 & 0.663 \\ \cline{2-10}
                                             & Precision & 0.514 & 0.521 & 0.519 & 0.514 & 0.516 & 0.514 & 0.521 & 0.524 \\ \cline{2-10}
                                             & Recall & 0.351 & 0.366 & 0.362 & 0.351 & 0.362 & 0.354 & 0.369 & 0.369 \\ \cline{2-10}
                                             & F1 & 0.417 & 0.43 & 0.426 & 0.417 & 0.425 & 0.419 & 0.432 & 0.433 \\ \hline  \hline


                                            \multirow{4}{*}{\textit{Equal-freq}}  & Accuracy & 0.727 & 0.728 & 0.734 & 0.734 & 0.736 & 0.734 & 0.733 & 0.725 \\ \cline{2-10}
                                             & Precision & 0.638 & 0.637 & 0.652 & 0.651 & 0.654 & 0.655 & 0.647 & 0.633 \\ \cline{2-10}
                                             & Recall & 0.5 & 0.511 & 0.511 & 0.515 & 0.515 & 0.504 & 0.519 & 0.507 \\ \cline{2-10}
                                             & F1 & 0.561 & 0.567 & 0.573 & 0.575 & 0.576 & 0.57 & 0.576 & 0.563 \\ \hline  \hline


                                            \multirow{4}{*}{\textit{CAIM}}  & Accuracy & 0.664 & 0.668 & 0.667 & 0.668 & 0.66 & 0.661 & 0.66 & 0.659 \\ \cline{2-10}
                                             & Precision & 0.581 & 0.614 & 0.615 & 0.61 & 0.566 & 0.574 & 0.561 & 0.556 \\ \cline{2-10}
                                             & Recall & 0.134 & 0.131 & 0.119 & 0.134 & 0.112 & 0.116 & 0.119 & 0.112 \\ \cline{2-10}
                                             & F1 & 0.218 & 0.215 & 0.2 & 0.22 & 0.187 & 0.193 & 0.197 & 0.186 \\ \hline  \hline
            \hline
        \end{tabular}
    \end{table}

\begin{figure}[H]
    \includegraphics[width=\textwidth]{img/conf_matrices/cm_Diabetes_cv9_KFold.png}
    \caption{Macierz konfuzji dla najlepszej wartości F1 -- kroswalidacja zwykła.}
\end{figure}
        \begin{figure}[H]
\center
    \includegraphics[width=\textwidth]{img/cv_scores_stratifiedkfold/scoring_stratifiedkfold_diabetes.png}
    \caption{Wykresy wartości metryk dla zbioru "Diabetes" -- kroswalidacja stratyfikowana.}
\end{figure}



    \pagebreak
    \subsubsection{Zbiór danych -- "Glass"}
        \subsection*{Kroswalidacja zwykła}

\begin{figure}[H]
\center
    \includegraphics[width=0.8\textwidth]{img/cv_scores_kfold/scoring_kfold_glass.png}
    \caption{Wartości metryk dla klasyfikatora Bayesowskiego.}
\end{figure}

\begin{figure}[H]
    \center
    \includegraphics[width=0.8\textwidth]{img/cv_plots/cv_ns_glass.png}
    \caption{Wartości metryk dla drzewa decyzyjnego C4.5.}
\end{figure}

\begin{table}[H]
  \center
\begin{tabular}{|c|c|c|c|c|c|c|c|c|c|}  \hline
  \multicolumn{2}{|c|}{} & \multicolumn{8}{c|}{Liczba foldów} \\ \hline
Konfiguracja &    Metryka &     2 &     3 &     4 &     5 &     6 &     7 &     8 &     9 \\ \hline
          C1 &   Accuracy &  0.67 &  0.64 &  0.65 &  0.72 &  0.70 &  0.65 &  0.73 &  0.65 \\ \hline
          C1 &  Precision &  0.64 &  0.64 &  0.64 &  0.76 &  0.74 &  0.67 &  0.72 &  0.67 \\ \hline
          C1 &     Recall &  0.66 &  0.62 &  0.61 &  0.69 &  0.71 &  0.65 &  0.66 &  0.64 \\ \hline
          C1 &         F1 &  0.63 &  0.65 &  0.67 &  0.71 &  0.76 &  0.74 &  0.77 &  0.70 \\ \hline \hline
          C2 &   Accuracy &  0.63 &  0.57 &  0.58 &  0.64 &  0.61 &  0.61 &  0.60 &  0.61 \\ \hline
          C2 &  Precision &  0.69 &  0.59 &  0.58 &  0.63 &  0.59 &  0.60 &  0.61 &  0.62 \\ \hline
          C2 &     Recall &  0.40 &  0.50 &  0.50 &  0.57 &  0.55 &  0.53 &  0.60 &  0.60 \\ \hline
          C2 &         F1 &  0.74 &  0.61 &  0.69 &  0.68 &  0.68 &  0.67 &  0.71 &  0.73 \\ \hline \hline
          C3 &   Accuracy &  0.54 &  0.65 &  0.63 &  0.63 &  0.64 &  0.65 &  0.62 &  0.70 \\ \hline
          C3 &  Precision &  0.57 &  0.66 &  0.63 &  0.59 &  0.58 &  0.62 &  0.65 &  0.68 \\ \hline
          C3 &     Recall &  0.41 &  0.57 &  0.57 &  0.67 &  0.60 &  0.69 &  0.67 &  0.69 \\ \hline
          C3 &         F1 &  0.61 &  0.70 &  0.67 &  0.66 &  0.70 &  0.69 &  0.70 &  0.74 \\ \hline
\end{tabular}
  \caption{Dokładne wartości metryk dla drzewa decyzyjnego C4.5.}
\end{table}

        
\begin{figure}[H]
\center
    \includegraphics[width=\textwidth]{img/cv_scores_stratifiedkfold/scoring_stratifiedkfold_glass.png}
    \caption{Wykresy wartości metryk dla zbioru "Glass" -- kroswalidacja stratyfikowana.}
\end{figure}



    \pagebreak
    \subsubsection{Zbiór danych -- "Wine"}
        \begin{figure}[H]
    \includegraphics[width=\textwidth]{img/cv_scores_kfold/scoring_kfold_wine.png}
    \caption{Wykresy wartości metryk dla zbioru "Wine" -- krowalidacja zwykła.}
\end{figure}

        \begin{table}[H]
        \center
        \caption{Wartości metryk dla zbioru "Wine" -- krowalidacja zwykła.}
        \begin{tabular}{|c|c|c|c|c|c|c|c|c|c|}
            \hline
            \multirow{2}{*}{\textbf{Metoda dyskr.}} & \multirow{2}{*}{\textbf{Metryka}} & \multicolumn{8}{|c|}{\textbf{CV}} \\ \cline{3-10}
                            &  & 2 & 3 & 4 & 5 & 6 & 7 & 8 & 9 \\ \hline
            \multirow{4}{*}{\textit{Brak}}  & Accuracy & 0.376 & 0.303 & 0.669 & 0.933 & 0.938 & 0.933 & 0.961 & 0.955 \\ \cline{2-10}
                                             & Precision & 0.13 & 0.114 & 0.657 & 0.934 & 0.938 & 0.934 & 0.959 & 0.954 \\ \cline{2-10}
                                             & Recall & 0.315 & 0.254 & 0.616 & 0.931 & 0.941 & 0.934 & 0.965 & 0.96 \\ \cline{2-10}
                                             & F1 & 0.184 & 0.157 & 0.574 & 0.932 & 0.939 & 0.934 & 0.962 & 0.957 \\ \hline \hline


                                            \multirow{4}{*}{\textit{Equal-width}}  & Accuracy & 0.303 & 0.124 & 0.489 & 0.792 & 0.77 & 0.798 & 0.843 & 0.837 \\ \cline{2-10}
                                             & Precision & 0.129 & 0.061 & 0.55 & 0.803 & 0.778 & 0.808 & 0.848 & 0.842 \\ \cline{2-10}
                                             & Recall & 0.254 & 0.103 & 0.467 & 0.801 & 0.783 & 0.809 & 0.853 & 0.85 \\ \cline{2-10}
                                             & F1 & 0.171 & 0.077 & 0.473 & 0.802 & 0.78 & 0.808 & 0.85 & 0.846 \\ \hline \hline


                                            \multirow{4}{*}{\textit{Equal-freq}}  & Accuracy & 0.348 & 0.197 & 0.663 & 0.848 & 0.837 & 0.837 & 0.871 & 0.865 \\ \cline{2-10}
                                             & Precision & 0.147 & 0.091 & 0.707 & 0.855 & 0.849 & 0.847 & 0.875 & 0.872 \\ \cline{2-10}
                                             & Recall & 0.291 & 0.164 & 0.656 & 0.86 & 0.853 & 0.852 & 0.882 & 0.878 \\ \cline{2-10}
                                             & F1 & 0.195 & 0.117 & 0.669 & 0.857 & 0.846 & 0.846 & 0.878 & 0.873 \\ \hline \hline


                                            \multirow{4}{*}{\textit{CAIM}}  & Accuracy & 0.343 & 0.208 & 0.764 & 0.854 & 0.865 & 0.871 & 0.882 & 0.882 \\ \cline{2-10}
                                             & Precision & 0.138 & 0.102 & 0.788 & 0.866 & 0.875 & 0.883 & 0.888 & 0.891 \\ \cline{2-10}
                                             & Recall & 0.286 & 0.174 & 0.766 & 0.863 & 0.875 & 0.884 & 0.895 & 0.893 \\ \cline{2-10}
                                             & F1 & 0.187 & 0.128 & 0.771 & 0.862 & 0.871 & 0.878 & 0.888 & 0.889 \\ \hline \hline

            \hline
        \end{tabular}
    \end{table}

\begin{figure}[H]
    \includegraphics[width=\textwidth]{img/conf_matrices/cm_Wine_cv8_KFold.png}
    \caption{Macierz konfuzji dla najlepszej wartości F1 -- kroswalidacja zwykła.}
\end{figure}
        \begin{figure}[H]
    \includegraphics[width=\textwidth]{img/cv_scores_stratifiedkfold/scoring_stratifiedkfold_wine.png}
    \caption{Wykresy wartości metryk dla zbioru "Wine" -- krowalidacja stratyfikowana.}
\end{figure}

\begin{table}[H]
\center
    \caption{}
    \begin{tabular}{|c|c|c|c|c|c|c|c|c|c|}
        \hline
        \multirow{2}{*}{\textbf{Metoda dyskr.}} & \multirow{2}{*}{\textbf{Metryka}} & \multicolumn{8}{|c|}{\textbf{CV}} \\ \cline{3-10}
                        &  & 2 & 3 & 4 & 5 & 6 & 7 & 8 & 9 \\ \hline
        \multirow{4}{*}{\textit{Brak}}  & Accuracy & 0.972 & 0.961 & 0.961 & 0.955 & 0.961 & 0.961 & 0.961 & 0.955 \\ \cline{2-10}
                                     & Precision & 0.973 & 0.961 & 0.959 & 0.954 & 0.96 & 0.959 & 0.959 & 0.954 \\ \cline{2-10}
                                     & Recall & 0.972 & 0.964 & 0.965 & 0.96 & 0.964 & 0.965 & 0.965 & 0.96 \\ \cline{2-10}
                                     & F1 & 0.972 & 0.962 & 0.962 & 0.957 & 0.962 & 0.962 & 0.962 & 0.957 \\ \cline{2-10}


        \multirow{4}{*}{\textit{Equal-width}}  & Accuracy & 0.899 & 0.899 & 0.904 & 0.904 & 0.893 & 0.899 & 0.904 & 0.899 \\ \cline{2-10}
                                             & Precision & 0.902 & 0.903 & 0.908 & 0.908 & 0.897 & 0.902 & 0.907 & 0.902 \\ \cline{2-10}
                                             & Recall & 0.91 & 0.909 & 0.914 & 0.914 & 0.904 & 0.908 & 0.913 & 0.908 \\ \cline{2-10}
                                             & F1 & 0.903 & 0.904 & 0.909 & 0.909 & 0.899 & 0.904 & 0.909 & 0.904 \\ \cline{2-10}


        \multirow{4}{*}{\textit{Equal-freq}}  & Accuracy & 0.893 & 0.888 & 0.91 & 0.899 & 0.904 & 0.91 & 0.916 & 0.916 \\ \cline{2-10}
                                             & Precision & 0.901 & 0.896 & 0.917 & 0.906 & 0.911 & 0.916 & 0.92 & 0.92 \\ \cline{2-10}
                                             & Recall & 0.905 & 0.898 & 0.921 & 0.908 & 0.915 & 0.92 & 0.924 & 0.924 \\ \cline{2-10}
                                             & F1 & 0.9 & 0.894 & 0.915 & 0.904 & 0.91 & 0.915 & 0.92 & 0.92 \\ \cline{2-10}


        \multirow{4}{*}{\textit{CAIM}}  & Accuracy & 0.916 & 0.899 & 0.904 & 0.916 & 0.899 & 0.91 & 0.904 & 0.904 \\ \cline{2-10}
                                         & Precision & 0.924 & 0.908 & 0.911 & 0.922 & 0.906 & 0.915 & 0.91 & 0.91 \\ \cline{2-10}
                                         & Recall & 0.926 & 0.911 & 0.917 & 0.928 & 0.91 & 0.92 & 0.915 & 0.915 \\ \cline{2-10}
                                         & F1 & 0.92 & 0.905 & 0.911 & 0.921 & 0.905 & 0.915 & 0.91 & 0.91 \\ \cline{2-10}

            \hline
    \end{tabular}
\end{table}

\begin{figure}[H]
    \includegraphics[width=\textwidth]{img/conf_matrices/cm_Wine_cv2_StratifiedKFold.png}
    \caption{Macierz konfuzji dla najlepszej wartości F1 -- kroswalidacja stratyfikowana.}
\end{figure}
